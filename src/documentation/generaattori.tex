\documentclass{article}
\usepackage[utf8x]{inputenc}
\usepackage[english]{babel}
\usepackage[hyphens]{url}
\usepackage{booktabs}
\usepackage{natbib}
\usepackage{hyperref}
\usepackage{listings}
\usepackage[margin=1in]{geometry}
\usepackage{amsthm}
\usepackage{centernot}
\usepackage{verbatim}
\bibliographystyle{plain}


\newtheorem{theorem}{Theorem}
\newtheorem{definition}{Definition}

\newcommand*{\fullref}[1]{\hyperref[{#1}]{\autoref*{#1} \nameref*{#1}}}

\title{\texttt{C++} code generation \\
{\small bootstrap strategy}}
\author{J Ulate C}
\date{\today}

\begin{document}

\maketitle

\section{Code generation}

In our journey as a programmer, we can find ourselves in a situation where we have a very repetitive operation. Sometimes we have data that has to be translated into code. Usually, the solution is to generate the boilerplate code using a tool; sometimes, we use to write a small script that does the job for us.

\subsection{Bootstrap}

Probably you have heard of the bootstrap paradox. Where an object or piece of information (a compiler in our case) is sent back in time and became trapped in a time loop. In this case, bootstrapping is about building a compiler using tools smaller than itself, as opposed to building a compiler using an already made version of itself.

\section{Example: Pretty printer for \texttt{C++} code}

This project present a very simple pretty-printer for \texttt{C++} code. The idea is to avoid to build a full \textit{Abstract Syntax Tree} with the \textit{emitter}, \textit{parser}, and all required machinery. This project presents a very simple pretty-printer for \texttt{C++} code. The idea is to avoid to build a full \textit{Abstract Syntax Tree} with the \textit{emitter}, \textit{parser}, and all required machinery. In this case, we will use a bootstrap strategy to build an enumeration of all reserved words and operator of the \texttt{C++} language and the printing method for each of them.

\subsection{Data preparation}

The first task that we performed is collecting the full list of all \textit{reserved words} and \textit{operators} of the Language. We compile them in \texttt{TSV} file. We defined a name, lexem, and a category for each word. You can take a look here.

\subsection{Compilation Unit}

\subsubsection{Identifiers}

Identifiers are the basic block of any programming language, apart from the \textit{operators} and \textit{reserved words}. In this example, we will handle local variables, public members, protected members,  private members, namespaces, and type names. We defined an enumeration for this purpose. Then we create a simple class that contains the \textit{lexem} and the category of the identifier. Of course, we wrote the operators and required constructors to be able to use them efficiently.

\subsubsection{Line information}

Another critical point is source code tracking. The particular fact about generated code is the source. The start a generated code is not the code itself, it is the generator's code. Given that fact, when we want to debug the code, we must prefer to follow the path in the code generator rather than the generated one. For this particular purpose, we carry the source file and line information using a straightforward class container.

\subsubsection{Compilation Unit}

The concept of a compilation unit is usually associated with a  single file that is processed by the compiler. In our implementation, we use an output stream instance where we write all the code, and we define pretty much all the language literal lexemes printing methods. Also, we handle the concept of Scopes using coding blocks taking advantage of the \textbf{RAII} design pattern. Using the capabilities of the language of creating anonymous lambda functions that handle each own \textit{Scope}, we create a mechanism that helps the readability of the code.

\subsubsection{Helper Macros}

We defined pre-processor macros that provide simplicity at the writing code time. In particular for the moment of open and closing \textit{Scopes}.


\subsection{\texttt{TSV} to \texttt{C++} translation}

The first step is simple, just parse the file and load all the rows in a \texttt{struct}. After that, we initialise the compilation unit, and we start writing the code. As any other header file, we define a guard, to avoid that the code is included more than once, later some library Include and the namespace declaration.

Then, we write the Enumeration as a declaration, and we traverse all the entries loaded from the file and print the corresponding entries.  Analogously, we apply the same procedure for the categories.

The most complicated part here is the print method when we declare a \texttt{template} method. We all know that \texttt{template}s in \texttt{C++} are detailed, now imagine a code generation for the \texttt{template} method. But, as any task, we start for the beginning, with the declaration, writing the prototype of the function. In the body, we add a colossal \texttt{switch} statement with all the entries of the file and the corresponding \texttt{case}, we print the lexem character by character.

Finally, we define a lookup array for the corresponding category of each \textit{reserved word} based in their position.

\section{Discussion}

As you may have noticed, we have previously generated the tokens header file, before we have compiled the code and rendered it. This is the trap in the time-travel loop.  Of course, where were previous iterations of this process, but they are not relevant to explain the concept of the bootstrap strategy.

As you may have noticed, we have previously generated the tokens header file, before we have compiled the code and rendered it. This is the trap in the time travel loop.  Of course, where were previous iterations of this process, but they are not relevant to explain the concept of the bootstrap strategy. 

In a posterior post, we will show how the code generation can be used to model a problem and the debugging process of the generator code. 

\end{document}